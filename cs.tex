\documentclass[10pt,landscape]{article}
\usepackage{multicol}
\usepackage{calc}
\usepackage{ifthen}
\usepackage[landscape]{geometry}
\usepackage{amsmath,amsthm,amsfonts,amssymb}
\usepackage{color,graphicx,overpic}
\usepackage{hyperref}



\pdfinfo{
  /Title (example.pdf)
  /Creator (TeX)
  /Producer (pdfTeX 1.40.0)
  /Author (Seamus)
  /Subject (Example)
  /Keywords (pdflatex, latex,pdftex,tex)}

% This sets page margins to .5 inch if using letter paper, and to 1cm
% if using A4 paper. (This probably isn't strictly necessary.)
% If using another size paper, use default 1cm margins.
\ifthenelse{\lengthtest { \paperwidth = 11in}}
    { \geometry{top=.5in,left=.5in,right=.5in,bottom=.5in} }
    {\ifthenelse{ \lengthtest{ \paperwidth = 297mm}}
        {\geometry{top=1cm,left=1cm,right=1cm,bottom=1cm} }
        {\geometry{top=1cm,left=1cm,right=1cm,bottom=1cm} }
    }

% Turn off header and footer
\pagestyle{empty}

% Redefine section commands to use less space
\makeatletter
\renewcommand{\section}{\@startsection{section}{1}{0mm}%
                                {-1ex plus -.5ex minus -.2ex}%
                                {0.5ex plus .2ex}%x
                                {\normalfont\large\bfseries}}
\renewcommand{\subsection}{\@startsection{subsection}{2}{0mm}%
                                {-1explus -.5ex minus -.2ex}%
                                {0.5ex plus .2ex}%
                                {\normalfont\normalsize\bfseries}}
\renewcommand{\subsubsection}{\@startsection{subsubsection}{3}{0mm}%
                                {-1ex plus -.5ex minus -.2ex}%
                                {1ex plus .2ex}%
                                {\normalfont\small\bfseries}}
\makeatother

% Define BibTeX command
\def\BibTeX{{\rm B\kern-.05em{\sc i\kern-.025em b}\kern-.08em
    T\kern-.1667em\lower.7ex\hbox{E}\kern-.125emX}}

% Don't print section numbers
\setcounter{secnumdepth}{0}


\setlength{\parindent}{0pt}
\setlength{\parskip}{0pt plus 0.5ex}

%My Environments
\newtheorem{example}[section]{Example}
% -----------------------------------------------------------------------

\begin{document}
\raggedright
\footnotesize
\begin{multicols}{3}


% multicol parameters
% These lengths are set only within the two main columns
%\setlength{\columnseprule}{0.25pt}
\setlength{\premulticols}{1pt}
\setlength{\postmulticols}{1pt}
\setlength{\multicolsep}{1pt}
\setlength{\columnsep}{2pt}

\begin{center}
     \Large{\underline{Complex Analysis Cheat Sheet}} \\
\end{center}

\section{Section 1}

\subsection{Complex Differentiation}
Let $\Omega_1$, $\Omega_2 \subset \mathbb{C}$ be open sets and let $f : \Omega_1 \rightarrow \Omega_2$. We say that f is differentiable (holomorphic) at $z_0 \in \Omega_1$ if the quotient $$\frac{f(z_0 + h) -  f(z_0)}{h}$$ converges to a limit when $h \rightarrow 0$. Here $h \in \mathbb{C}$, $h \neq 0$ and $z_0 + h \in \Omega_1$. The limit of this quotient, when it exists, is denoted by $f'(z_0)$, and is called the
derivative of $f$ at $z_0$:$$f'(z_0) = \lim_{h \rightarrow 0} \frac{f(z_0 + h) - f(z_0)}{h}.$$

The function $f$ is said to be \textit{holomorphic on an open set}  $\Omega$ if $f$ is
holomorphic at every point of $\Omega$.

If $C$ is a closed subset of $\mathbb{C}$, we say that $f$ is holomorphic on $C$ if $f$ is
holomorphic in some open set containing $C$. Finally, if $f$ is holomorphic in
all of $\mathbb{C}$ we say that $f$ is \textit{entire}.

\textbf{Corollary:} If a function $f$ is holomorphic then it is continuous.
\subsection{Properties of the complex derivative}
If $f$ and $g$ are holomorphic in $\Omega$ then:
\begin{enumerate}
\item $f + g$ is holomorphic in $\Omega$ and $(f + g)' = f' + g'.$\\
\item $fg$ is holomorphic in $\Omega$ and $(fg)' = f'g + fg'.$\\
\item $g(z_0) \neq 0$, then $f/g$ is holomorphic at $z_0$ and $$(f / g)' = \frac{f'g - fg'}{g^2}$$\\
\item Moreover, if $f : \Omega \rightarrow U$ and $g : U \rightarrow \mathbb{C}$ are holomorphic, the chain rule holds:
$$(g \circ f)(z) = g'(f(z))f'(z), \forall z \in \Omega.$$
\end{enumerate}
\subsection{Cauchy-Riemann equations}
The functions $u$ and $v$ satisfy the following:
$$\frac{\partial u}{\partial x} = \frac{\partial v}{\partial y} \, \text{ and } \, \frac{\partial u}{\partial y} = -\frac{\partial v}{\partial x}.$$
\textbf{Remark:} The Cauchy-Riemann equations link real and complex analysis.
$$\frac{\partial}{\partial z} = \frac{1}{2}\left(\frac{\partial }{\partial x} + \frac{1}{i} \frac{\partial}{\partial y}\right) \, \text{ and } \, \frac{\partial}{\partial \overline{z}} = \frac{1}{2}\left(\frac{\partial }{\partial x} - \frac{1}{i} \frac{\partial}{\partial y}\right).$$
\textbf{Theorem:} Let $f(z) = u(x, y) + iv(x, y)$, $z = x + iy$. If $f$ is holomorphic
at $z_0$, then
$$\frac{\partial f}{\partial \overline{z}}(z_0) = 0 \, \text{ and } \, f'(z_0) = \frac{\partial f}{\partial z}(z_0) = 2\frac{\partial u}{\partial z}(z_0).$$

\textbf{Theorem:} Suppose $f = u + iv$ is a complex-valued function defined on
an open set $\Omega$. If $u$ and $v$ are continuously differentiable and satisfy the
Cauchy-Riemann equations on $\Omega$, then $f$ is holomorphic on $\Omega$ and $f'(z) = \partial f(z)/ \partial z$.
\subsection{Harmonic functions}
\textbf{Definition:} Let $\varphi = \varphi(x, y)$, $x,y \in \mathbb{R}$ be a real function of two variables.
$\varphi$ is said to be \textit{harmonic} in an open set $\Omega \in \mathbb{R}^2$ if
$$\Delta \varphi(x, y) := \frac{\partial^2 \varphi}{\partial x^2}(x,y) + \frac{\partial^2 \varphi}{\partial y^2}(x,y) = \varphi''_{xx}(x,y) + \varphi''_{yy}(x,y) = 0.$$
\textbf{Note:} Usually $\Delta$ is called the Laplace operator.

\textbf{Theorem:} Let $f(z) = u(x, y) + iv(x, y)$ be holomorphic in an open set
$\Omega \subset \mathbb{C}$. Then $u$ and $v$ are harmonic. For the proof, use the Cauchy-Riemann equations.

\textbf{Theorem (Harmonic Conjugate):} Let $u$ be harmonic in an open set $\Omega \subset \mathbb{C}$. Then there exists a
harmonic function $v$ such that $f = u + iv$ is holomorphic in $\Omega$. In this case
$v$ is called the \textit{harmonic conjugate} to $u$.

\textbf{Theorem:} Assume that $f = u + iv$ is a holomorphic function defined on
an open connected domain $\Omega \subset \mathbb{C}$. Consider the two equations
$$a) \quad u(x, y) = C \qquad \text{and} \qquad b) \quad v(x, y) = K,$$ where $C, K$ are two real constants.
Assume that the equations $a)$ and $b)$ have the same solution $(x_0, y_0)$ and
that $f'(z_0) \neq 0$ at $z_0 = x_0 + iy_0.$ Then the curve defined by the equation $a)$
is \textit{orthogonal} to the curve defined by the equation $b)$ at $(x_0, y_0)$.

\textit{Proof.} Use the Cauchy-Riemann equations to prove that the vectors $\nabla u$ and $\nabla v$ are orthogonal at $(x_0,y_0)$.


\subsection{Cauchy-Riemann equations in polar coordinates.}
For a holomorphic function $f = u + iv$, introduce
$$x = r \cos \theta, \quad y = r \sin \theta, \quad r =
\sqrt{x^2 + y^2}, \quad \theta = \arctan y/x.$$
Then
$$u'_r = \frac{1}{r}v'_{\theta} \quad \text{and similarly} \quad v'_r  = - \frac{1}{r}u'_{\theta}.$$
\subsection{Logarithmic functions}
Let $z = r(cos \theta + isin \theta) = r e^{i\theta}$.

\textbf{Definition:}
$\log z = \ln |z| + i \arg z = \log r + i(\theta + 2 \pi k),$ $z \neq 0, $ where $ k = 0, \pm1, \pm2, \ldots$

\textbf{Remark.} The function $\log$ is a multi-valued function.

\textbf{Definition:} We define $\text{Log }  z$ as the single-valued function:
$$\text{Log } z = \ln |z| + i \text{Arg } z,$$
where $\text{Arg } z$ is the \textit{principal value} of the argument, namely, $−\pi < \text{Arg } z \leq
\pi$.

\textbf{Definition (Branch cut):}
A curve (with ends possibly open, closed, or
half-open) in the complex plane across which a holomorphic multivalued
function is \textit{discontinuous}.

\textbf{Example:} $\text{Log } z$ has its branch cut defined by  $(- \infty, 0]$.
\subsection{Integration along curves}
For the integration of a continuous function over a curve.

\textit{ML Inequality:} $|\int_{\gamma}f(z) \,\text{d}z| \leq \sup_{z \in \gamma} |f(z)| \cdot \text{length} (\gamma).$

\textbf{Definition:} A \textit{primitive} for $f$ on $\Omega \subset \mathbb{C}$ is a function $F$ that is holomorphic
on $\Omega$ and such that $F'(z) = f(z)$ for all $z \in \Omega$.

\textbf{Theorem:} If a continuous function $f$ has a primitive $F$ in an open set $\Omega$, and $\gamma$ is a curve in $\Omega$ that begins at $w_1$ and ends at $w_2$, then
$$\int_{\gamma} f(z)\, \text{d}z = F(w_2) - F(w_1).$$

\textbf{Corollary:} If $\gamma$ is a closed curve in an open set $\Omega$, $f$ is continuous and
has a primitive in $\Omega$, then $$\int_{\gamma} f(z)\, \text{d}z = 0.$$
This is immediate since the end-points of a closed curve coincide.

\textbf{Corollary:} If $f$ is holomorphic in an open connected set $\Omega$ and $f' = 0$,
then $f$ is constant.

\textbf{Theorem (Green's theorem):} Suppose $P(x, y)$ and $Q(x, y)$ have continuous
first partial derivatives in an open set $\widetilde{\Omega}$ containing a simple, closed,
piecewise-smooth curve $\gamma$ and its interior $\Omega \subset \widetilde{\Omega}.$ Then $$\oint_{\gamma} P \, \text{d}x + Q \, \text{d} y = \iint_{\Omega} \left(\frac{\partial Q}{\partial x} - \frac{\partial P}{\partial y}\right) \, \text{d}x\, \text{d}y.$$


\subsection{Cauchy's integral formulas}

\textbf{Theorem (Cauchy-Goursat):}
Let $f$ be holomorphic inside $\Omega$
bounded by a closed piecewise-smooth simple curve $\gamma$ and also at the points
of $\gamma$. Then
$$\oint_{\gamma} f(z) \, \text{d}z = 0.$$

\textbf{Theorem (Deformation):}
Let $\gamma_1$ and $\gamma_2$ be two simple, closed,
piecewise-smooth curves with $\gamma_2$ lying wholly inside $\gamma_1$ and suppose $f$ is
holomorphic in a domain containing the region between $\gamma_1$ and $\gamma_2$. Then
$$\int_{\gamma_1} f(z) \, \text{d}z = \int_{\gamma_2} f(z) \, \text{d}z  $$

\textbf{Theorem (Cauchy’s integral formulae):}
Let $f$ be holomorphic inside and on a simple, closed,
piecewise-smooth curve $\gamma$. Then for any point $z_0$ interior to $\gamma$ we have
$$f(z_0) = \frac{1}{2 \pi i} \oint_{\gamma} \frac{f(z)}{z - z_0} \, \text{d}z.$$

\textbf{Theorem:} Let $f$ be holomorphic in an open set $\Omega$, then $f$ has \textit{infinitely}
many complex derivatives in $\Omega$. Moreover, for a simple, closed, piecewise-smooth
curve $\gamma \subset \Omega$ and any $z$ lying inside $\gamma$ we have
$$\frac{\text{d}^n f(z)}{\text{d}z^n} = \frac{n!}{2 \pi i} \oint_{\gamma} \frac{f(\eta)}{(\eta - z)^{n+1}} \, \text{d}\eta$$

\textbf{Corollary:} If $f$ is holomorphic in $\Omega$, then all its derivatives $f',f'', \ldots $ are holomorphic.

\textbf{Theorem (Liouville's):} If an entire function is bounded, then it
is constant.

\textbf{Theorem (Maximum Modulus Principle):}  If $f$ is holomorphic and not
constant in an open connected set $\Omega$, then $|f(z)|$ \textit{cannot} take on a maximum value in $\Omega$.

\textbf{Corollary:} If $f$ is holomorphic in a bounded open connected set and
continuous on its boundary $\partial \Omega$, then $|f(z)|$ \textit{attains its maximum} on $\partial \Omega$.


\subsection{Laurent Series}
\textbf{Definition:} We say that $f$ has a \textit{zero of order m} at $z_0 \in \mathbb{C}$ if $$f^{(k)}(z_0) = 0, \quad k = 0,1,\ldots m-1,$$ and $f^{(m)}(z_0) \neq 0.$

\textbf{Theorem} A holomorphic function $f$ has a zero of order $m$ at $z_0$ if and
only if it can be written in the form
$$f(z) = (z-z_0)^mg(z),$$
where $g$ is holomorphic at $z_0$ and $g(z_0) \neq 0$.

\textbf{Corollary:} The zeros of a non-constant holomorphic function are isolated;
that is every zero has a neighbourhood inside of which \textit{it is the only
zero}. 

\textbf{Theorem (Laurent Expansion):} Let $f$ be holomorphic in the
annulus $D = \{z : r < |z - z_0| < R\}.$ Then $f(z)$ can be expressed in the
form $$\sum_{n = -\infty}^{\infty} a_n (z - z_0)^n,$$ where $$a_n = \frac{1}{2 \pi i} \oint_{\gamma} \frac{f(\eta)}{(\eta - z_0)^{n+1}} \, \text{d}\eta,$$ and where $\gamma$ is any simple, closed, piecewise-smooth curve in $D$ that contains
$z_0$ in its interior.

\subsection{Singularities}
\textbf{Definition:} A point $z_0$ is called a \textit{singularity} of a complex function $f$ if $f$
is not holomorphic at $z_0$, but every neighbourhood of $z_0$ contains at least
one point at which $f$ is holomorphic.

\textbf{Definition:} A singularity $z_0$ of a complex function is said to be \textit{isolated} if
there exists a neighbourhood of $z_0$ in which $z_0$ is the only singularity of $f$.

\textbf{Definition:} Suppose a holomorphic function $f$ has an isolated singularity
at $z_0$ and $$f(z)= \sum_{n = -\infty}^{\infty} a_n (z - z_0)^n$$ is the Laurent expansion of $f$ valid in some annulus $0 < |z - z_0| < R.$ Then
\begin{itemize}
\item if $a_n = 0$ for all $n < 0$, $z_0$ is called a \textit{removable} singularity.
\item if $a_n = 0$ for $n < -m$ where $m$ is a fixed positive integer, but $a_{-m} \neq 0,$ $z_0$ is called a \textit{pole of order} $m$.
\item If $a_n \neq 0$ for \textit{infinitely many} negative $n$'s, $z_0$ is called an \textit{essential}
singularity.
\end{itemize}

\textbf{Theorem:} A function $f$ has a pole of order $m$ at $z_0$ if and only if it can be
written in the form
$$f(z) = \frac{g(z)}{(z - z_0)^m}
,$$
where $g$ is holomorphic at $z_0$ and $g(z_0) \neq 0$.

\subsection{Residue Theory}
\textbf{Definition:} Given the Laurent series for a function $f$ at $z_0$, the \textit{residue} of $f$ at $z_0$ is
$$\text{Res} [f, z_0] = a_{-1}.$$
\textbf{Theorem:} Let $f$ be a holomorphic function inside and on a simple, closed,
piecewise-smooth curve $\gamma$ except at the singularities $z_1, \ldots , z_n$ in its interior
$\Omega$. Then $$\oint_{\gamma} f(z) \, \text{d}z = 2 \pi i \sum_{j=1}^n \text{Res}[f,z_j].$$
Let $$f(z) = a_{-m}(z - z_0)^{-m} + a_{-m+1}(z - z_0)^{-m+1}+ \ldots$$ Then $$\text{Res} [f, z_0] = \lim_{z \rightarrow z_0} \frac{1}{(m-1)!}\frac{\text{d}^{m-1}}{\text{d}z^{m-1}}\left( (z - z_0)^m f(z) \right)$$
% You can even have references
\rule{0.3\linewidth}{0.25pt}
\scriptsize
\bibliographystyle{abstract}
\bibliography{refFile}
\end{multicols}
\end{document}
